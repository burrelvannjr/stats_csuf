\documentclass{article}
\usepackage[margin=1.0in]{geometry}
\usepackage{authblk} %package for blocking authors, followed by blocking affiliation
\usepackage{url}
\usepackage{ulem} % when using ulem package, must change \emph to \it for italics.
\usepackage{hyperref}
\usepackage{array}
\usepackage{amssymb,amsmath,tabu}
\usepackage{hyperref}
\usepackage[super]{nth}
\setlength\parindent{0pt}
\usepackage[english]{babel}


\begin{document}
\title{Homework 2\\ SOCI 303: Statistics for the Social Sciences (Spring 2017) \\ {\large{10 points}}}
\author[*]{}
\date{}
\maketitle



\section*{Overview:}
In this assignment, you'll be completing 3 problems using R. These problems cover topics from Chapters 6 through 8 in your textbook. \textbf{Remember to append/copy and paste your R script as the final page of this assignment} (as a new page).

\subsection*{Problem 1 (\textit{The Normal Distribution})}
\begin{itemize}
\item Using any mean and standard deviation, plot a normal distribution using $N=100$ cases. Be sure to show your plot.
\item Using any mean and standard deviation, plot a normal distribution using $N=10,000$ cases. Be sure to show your plot.
\item Comment on the differences between the two plots.
\end{itemize}

\subsection*{Problem 2 (\textit{Sampling})}
\begin{itemize}
\item Using the \texttt{USArrests} data set in R, take 3 random samples of $N=20$. Remember to give each sample a unique object name so you can do subsequent calculations.
\item Calculate the means and standard deviations for the \texttt{Murder} variable for each sample.
\item Comment on the differences between the 3 means, and on the differences between the 3 standard deviations. Do they differ? Why or why not?
\end{itemize}

\subsection*{Problem 3 (\textit{Estimation})}
\begin{itemize}
\item Report the following formulas: 
	\begin{itemize}
	\item Z-score
	\item 99\% confidence interval
	\item Standard error of the mean of the sampling distribution ($S_{\bar{Y}}$)
	\end{itemize}
\item Using the \texttt{USArrests} data set in R, calculate the mean, standard deviation, and the sampling error of the mean of the sampling distribution for the \texttt{Assault} variable. Do not save these as new variables in the data set.
\item Calculate the z-scores for the \texttt{Assault} variable for all observations and save this as a new variable in the data set. Report the z-score value (of the \texttt{Assault} variable) for the \nth{12} observation.
\item Calculate the 99\% confidence interval for the \texttt{Assault} variable. Report the lower and upper bounds/values of this confidence interval. 
\end{itemize}


\end{document}











