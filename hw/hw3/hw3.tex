\documentclass{article}
\usepackage[margin=1.0in]{geometry}
\usepackage{authblk} %package for blocking authors, followed by blocking affiliation
\usepackage{url}
\usepackage{ulem} % when using ulem package, must change \emph to \it for italics.
\usepackage{hyperref}
\usepackage{array}
\usepackage{amssymb,amsmath,tabu}
\usepackage{hyperref}
\usepackage[super]{nth}
\setlength\parindent{0pt}
\usepackage[english]{babel}


\begin{document}
\title{Homework 3\\ SOCI 303: Statistics for the Social Sciences (Spring 2017) \\ {\large{10 points}} \\ {\large{Due: May 5, 2017}}}
\author[*]{}
\date{}
\maketitle



\section*{Overview:}
In this assignment, you'll be completing 1 problem using R. This problem covers topics from Chapter 11 in your textbook. \textbf{Remember to append/copy and paste your R script as the final page of this assignment} (as a new page).

\subsection*{Problem 1 (\textit{Chi-Sqaure Test of Independence})}
\begin{itemize}
\item Using the \texttt{surveys} data set in R (from the \texttt{MASS} package), report a bivariate table for the relationship between the sex variable (\texttt{Sex}) and the smoking variable (\texttt{Smoke}).
\item Given the number of rows and columns in your table, report the degrees of freedom for the bivariate relationship between the sex variable (\texttt{Sex}) and the smoking variable (\texttt{Smoke}). 
\item What is the value of the Chi-Square statistic? 
\item Correctly and fully report and interpret the Chi-Square statistic (using the example from our lecture slides, including the $X^2$ value, the degrees of freedom, and the p-value) 
\end{itemize}

\end{document}











