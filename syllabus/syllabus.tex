\documentclass{article}
\usepackage[margin=1.0in]{geometry} %change all margins to 1.0 inches (except the title, but moves up)
%\documentclass[12pt]{article}
%\usepackage[margin=.8in]{geometry} 
%\usepackage{authblk} %package for blocking authors, followed by blocking affiliation
\usepackage{url}
\usepackage{ulem} % when using ulem package, must change \emph to \it for italics.
\usepackage{hyperref} %allow for hyper links
\usepackage [english]{babel}
\usepackage [autostyle, english = american]{csquotes}
\MakeOuterQuote{"}
\usepackage[T1]{fontenc} %add encoding for small caps
\def\changemargin#1#2{\list{}{\rightmargin#2\leftmargin#1}\item[]}
\let\endchangemargin=\endlist 
\usepackage{titling} %title margin editing
\setlength{\droptitle}{-.75in} %size of top margin
\usepackage{setspace}
\usepackage{changepage} %changes margins using adjustwidth
\usepackage{tabularx}
\usepackage{tabu}
\usepackage{longtable}
\usepackage[super]{nth}
\usepackage{paralist} %to use compact item stuff
\makeatletter
\newcommand\tabfill[1]{%
\dimen@\linewidth%
\advance\dimen@\@totalleftmargin%
\advance\dimen@-\dimen\@curtab%
\parbox[t]\dimen@{\raggedright #1\ifhmode\strut\fi}%
}
%%%%below changes footer to special footer with name and page number
\usepackage{fancyhdr}
\pagestyle{fancy} %can be {fancy}
\cfoot{\thepage}
\renewcommand{\headrulewidth}{0pt}
%%%%above changes footer to special footer with name and page number
\begin{document}
%\date{today}
%\maketitle

\begingroup  
  \centering
  \begin{spacing}{1.5} %begins 1.5 spacing
  \textsc{\textbf{\LARGE{Statistics for the Social Sciences}}} %textsc is small caps, textbf is bold font, huge is largest font possible
  \end{spacing}
  \begin{spacing}{1.0} %begins single-spacing
  \centerline{\large Sociology 303}
  \centerline{\large Spring 2017}
  %\centerline{\normalsize bvann@uci.edu \textbullet \space (714) 398-5815 \textbullet \space \href{http://www.burrelvannjr.com}{burrelvannjr.com}}
  \end{spacing}
\endgroup
\raggedright %left-justifies text AKA does not justify all of text


%\begingroup %date group start
  %��\centerline{} %line space
  %\centerline{} %line space
   %\centerline{( {\it{\today}} )} %today's date, italicized with parentheses
%\endgroup %end date group

%\begin{singlespace}
Time: \textbf{F: 1:00pm-3:45pm} \hfill  \hfill Instructor: \textbf{Burrel Vann Jr} \\
Room: \textbf{IRVC--203} \hfill  \hfill Email: \textbf{bjvann@fullerton.edu} \\
Website: \textbf{\href{https://moodle-2016-2017.fullerton.edu/course/view.php?id=78248}{SOCI 303}} \hfill  \hfill Office: \textbf{IRVC--261} \\
  \hfill  \hfill Office Hours: \textbf{F: 12:00pm--1:00pm} \\
%\end{singlespace}


%\begin{singlespace}
\section*{Course Description}
Statistics is a ``language'' that can be used to describe social phenomena. When we talk about how things vary or relate to one another, we are talking about the relationship between two or more aspects of society. In quantitative (statistical) work, these aspects are known as variables. This course will provide students with the skills necessary for understanding, interpreting and drawing conclusions from statistical analysis of data. We will cover univariate and bivariate statistics, including probability and the normal curve; measures of central tendency, variation/dispersion, and confidence intervals; comparing means and proportions for two groups (t-tests); comparing means for more than two groups (ANOVA); correlation, and regression. Given the growing use of open-source programs, and the increasing demand for programming skills, students will conduct statistical analyses by hand and in the program R/RStudio.
%%\end{singlespace}

%\begin{singlespace}
\section*{Course Objectives}
\begin{itemize}
\item To understand the application of statistics to quantitative data to answer social science questions.\vspace*{-.75em}
\item To gain statistical analysis skills using R/RStudio.\vspace*{-.75em}
\item To develop programming skills that are transferrable to other statistical software platforms (including STATA, Python, SAS, and SPSS). \vspace*{-.75em}
\item To be able to interpret statistical results and clearly communicate conclusions
\end{itemize}


\section*{Course Materials}
\subsection*{Required}
\textbf{Textbook} \newline
Chava Frankfort-Nachmias and Anna Leon-Guerrero. 2015. \textit{Social Statistics for a Diverse Society}. Seventh Edition. Thousand Oaks, CA: SAGE Publications, Inc. \newline

\textbf{Calculator} \newline
You will need a basic calculator that can do basic functions (including add, subtract, multiply, divide, and square root). It does not need to be a graphing calculator. You will want to bring both your book and calculator to class everyday. \newline

\textbf{R/RStudio} \newline
Since we will be using the R statistical software package and its graphical user interface (GUI) RStudio, you will be required to download the program on lab computers. If you want to do your assignments at home, you can download R and RStudio on your own computer.

\section*{Course Requirements}
Students are required to attend all class meetings and participate in discussions, turn in homework assignments and in-class assessments, and complete three exams.\newline

\textbf{Participation (50 points):} \newline
Your participation grade is dependent upon your attendance and participation in in-class assessments. Attendance for this class is critical for your overall success in the course. If you miss a class meeting, look on the course website for material you may have missed. Second, if you find it difficult to understand some of the material, get in contact with your one or more of your classmates via Titanium. Third, if you still find it difficult, set aside time to meet with me in office hours. If my office hours don't work, email me so that we can schedule a time to meet. I reserve the right to re-do a lecture. In-class assessments will vary and can consist of short quizzes or R/RStudio output. \newline

\textbf{Homework (50 Points, 10 points each)} \newline
This class includes five homework assignments designed to help you grasp statistical methods. Nowadays, most statistical analyses are conducted using software. Therefore, most homework assignments will consist of analyses conducted in R/RStudio (some portion of each homework assignment may include hand calculations). Homework assignments are due at the beginning of class (by 1:15PM) on the day they are due. Any assignment turned in after 1:15PM is considered late. Please refer to my policy on late homework below. \newline

\textbf{Exams (150 points, 50 points each)} \newline
There will be three exams during the semester. Exams are closed book but you may use one page (8 1?2 inch by 11 inch), front and back, of notes (AKA ``cheat sheet''). For each exam, you will need to bring a calculator. All exams will be electronic, and must be taken in class in the computer lab. If you will miss an exam, you must inform the instructor before the start of the exam. Make up exams are not guaranteed and will be dealt with on a case-by-case basis. The final exam (Exam 3) will take place on May 19th, 2:30PM-4:20PM. \newline

\textbf{Policy on Late Homework and Make-Up Exams} \newline
Homework is considered late after the beginning of class on the day it is due (considered late after 1:15PM). Late homework will only be accepted for a maximum of seven days after the original due date, and will be granted a maximum of half credit. \newline

Students are not guaranteed make-up exams. Arrangements to take an exam early may be made. In extreme emergencies, written documentation will be required before a make-up exam is scheduled. In such cases, students will take a different and likely more difficult form of the exam. \newline

\textbf{Extra Credit} \newline
Students may be given the opportunity to complete one additional homework assignment for extra credit, worth a maximum of 10 points. If granted, extra credit is only accepted when it is due and will not be accepted late. 

\section*{Grading Breakdown}
Final grades will be based on five homework assignments, three exams (40 points each), and participation (attendance and in-class assessments) for a total of 250 points. A +/- grading system will not be used.

\begin{tabbing}
\quad \quad \quad \= Participation \quad \quad \= \tabfill{50 (attendance and in-class assessments)}\\
\> Homework \> \tabfill{50 (5 assignments, 10 points each)}\\
\> Exam 1 \> \tabfill{50}\\
\> Exam 2 \> \tabfill{50}\\
\> Exam 3 \> \tabfill{50}\\
\> Total  \> \tabfill{250}
\end{tabbing}

\textbf{Letter Grades}
\vspace*{-.5em}
\begin{tabbing}
\quad \quad \quad \= A = 90\% and above \\
\> B = 80\% and above \\
\> C = 70\% and above \\
\> D = 60\% and above \\
\> F = Below 60\% \\
\end{tabbing}

\section*{Classroom Conduct}
Please be courteous to your classmates and me by remaining engaged and respectful. Students are expected to conduct themselves in a way that does not interfere with the educational experience of others. Additionally, turn cell phones and other electronic devices on silent during class time. Laptops may be used for taking notes or running analyses while in class.


\section*{Academic Dishonesty}
The California State University, Fullerton policy on academic integrity is explained in \href{http://www.fullerton.edu/senate/publications_policies_resolutions/ups/UPS%20300/UPS%20300.021.pdf}{University Policy Statement 300.021}. All work you turn in, including homework assignments, exams, and quizzes must be your own.

\section*{Students with Special Needs}
Please inform the instructor during the first week of classes about any disability or special needs that you may have that may require specific arrangements related to attending class sessions, carrying out class assignments, or writing papers or examinations. According to California State University policy, students with disabilities must document their disabilities at the Disability Support Services (DSS) Office in order to be accommodated in their courses. Additional information can be found at the \href{http://www.fullerton.edu/dss/}{DSS website}, by calling 657-278-3112, or by email at dsservices@fullerton.edu.

\section*{Changes to Material}
I reserve the right to make changes to the syllabus, including the course outline, at any time, based on the pace of the class.


\newpage







\section*{Course Schedule}

\subsubsection*{1 - \textit{Introduction to Statistics}}
\begin{itemize}
\item \textbf{Chapter(s)}: 1
\end{itemize}

\vspace{3pt}

\subsubsection*{2 - \textit{Frequency Distributions; Graphic Presentation of Data}}
\begin{itemize}
\item \textbf{Chapter(s)}: 2 \& 3
\end{itemize}

\vspace{3pt}

\subsubsection*{3 - \textit{Measures of Central Tendency}}
\begin{itemize}
\item \textbf{Chapter(s)}: 4 
\end{itemize}

\vspace{3pt}

\subsubsection*{4 - \textit{Measures of Variability/Dispersion}}
\begin{itemize}
\item \textbf{Chapter(s)}: 5
\end{itemize}

\vspace{3pt}

\subsubsection*{5 - \textit{Review; ``Bringing Race Home'' Exercise; Introduction to R}}
\begin{itemize}
\item \textbf{Exercise}: Explore \textbf{\href{https://demographics.virginia.edu/DotMap/}{Racial Dot Map}}
\item \textbf{Due}: Download R \& RStudio
\item \textbf{Due}: HW 1 (Ch. 1, 2, 3, 4, 5)
\end{itemize}

\vspace{3pt}

\subsubsection*{6 - Exam 1 \newline}
\vspace{3pt}


\subsubsection*{7 - \textit{Normal Distribution}}
\begin{itemize}
\item \textbf{Chapter(s)}: 6
\end{itemize}

\vspace{3pt}

\subsubsection*{8 - \textit{Sampling Distributions}}
\begin{itemize}
\item \textbf{Chapter(s)}: 7
\end{itemize}

\subsubsection*{9 - \textit{Estimation}}
\begin{itemize}
\item \textbf{Chapter(s)}: 8
\end{itemize}

\vspace{3pt}

\subsubsection*{10 - NO CLASS: Spring Break \newline}

\vspace{3pt}


\subsubsection*{11 - \textit{Review}}
\begin{itemize}
\item \textbf{Due}: HW 2 (Chapters 6, 7, 8)
\end{itemize}

\vspace{3pt}

\subsubsection*{12 - Exam 2 \newline}

\vspace{3pt}

\subsubsection*{13 - \textit{Bivariate Tables; Chi Square}}
\begin{itemize}
\item \textbf{Chapter(s)}: 10 \& 11
\end{itemize}

\vspace{3pt}

\subsubsection*{14 - \textit{Hypothesis Testing (t-test)}}
\begin{itemize}
\item \textbf{Chapter(s)}: 9  
\item \textbf{Due}: HW 3 (Chapter 11)
\end{itemize}

\vspace{3pt}

\subsubsection*{15 - \textit{Analysis of Variance (ANOVA)}} 
\begin{itemize}
\item \textbf{Chapter(s)}: 12
\end{itemize}

\vspace{3pt}

\subsubsection*{16 - \textit{Correlation \& Regression}}
\begin{itemize}
\item \textbf{Chapter(s)}: 13  
\item \textbf{Due}: HW 4 (Chapters 9, 12)
\end{itemize}

\vspace{3pt}

\subsubsection*{Finals Week - Exam 3 (2:30pm--4:20pm)}
\begin{itemize}
\item \textbf{Due}: HW 5 (Chapter 13)
\end{itemize}








\end{document}